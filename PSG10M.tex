\documentclass[11pt]{article}

%\usepackage{sectsty}
\usepackage{siunitx}
\usepackage{tabularx}
\usepackage{float}
\usepackage{graphicx}
\usepackage{mathrsfs}
\usepackage{subcaption}
\usepackage{hyperref}
\usepackage{url}
\usepackage{csquotes}
\usepackage{verbatim}
\usepackage{cite}
\usepackage{stfloats}
\usepackage{textcomp}
\usepackage{algorithm}
\usepackage{algorithmic}
\usepackage{amsmath, amsfonts}
\usepackage{cmsrb}
\usepackage[serbian]{babel}

% Margins
\topmargin=-0.45in
\evensidemargin=0in
\oddsidemargin=0in
\textwidth=6.5in
\textheight=9.0in
\headsep=0.25in

\title{PSG10M}

\begin{document}

\maketitle
\pagebreak
\tableofcontents
\pagebreak

\section{Power supply}
Power is received via a USB C connector and negociated 
to 5V at 3A using 5.1k resistors on CC lines.

This voltage is boosted to 20V and all other voltage 
levels derive form this one. 

15 V rail is loosely regulated using a capacitance
multiplier. 5V rail is generated using a 7805 IC.

-20 V are generated by inverting the aforementioned 20 V
rail. -15 V and -5V rails are obtained the same way as 
their positive counterparts.

\subsection{Expected load}
Worst case load is 10V into 50 ohms, yielding 200 mA 
of current. This current is however limited by the 
output buffer to 150 mA. 

Say 50 mA for all operational amplifiers, and 50 mA for 
the MCU and the display. Total current capacity of the 
capacitance multiplier BJTs has to be 250 mA.

At 5V drop, this results ing 1.25W of waste heat. 
That's too much.

\subsection{Revision 1}
150 mA is fixed, but there is no need to power the 
LCD and the MCU with clean power. Only the AD DDS chip.

Further reducing the voltage drop to around 3V, yields 
a loss of 0.45 W. Bordering on manageable, considering 
this is an edge case only occurring on one voltage 
multiplier at a time.

5V still gets generated by 7x05 family, while the 
MCU and LCD get direct VBUS power.


\pagebreak
\section{User interface}
\subsection{16x2 LCD Display}
First row displays currently modified parameter: either freqeuncy, amplitude, 
dc offset, signal shape, duty cycle.
Second row lists those parameters: 
Freq Amp DC Sig Duty On/Off ... with scroll functionality.

\subsubsection{Display mounting}
Soldering on the front panel PCB directly by removing the I2C converter, 
and pouring solder in the mounting holes to attach to front panels back side.

Other option is mounting it using a screw terminal and its 3 mm screw.

\subsection{Rotary encoder}
First rotary encoder "Value encoder" modifies the currently active parameter.
When amplitude is selected in the bottom row "menu row", than rotating the 
value encoder changes amplitude.

Second rotary encoder, "menu encoder", changes the currently active parameter
in the menu row.

\subsubsection{Frequency menu}
Shows current frequency when non PWM mode. If PWM, shows duty cycle as well.
Pressing menu encoder shows period instead. If PWM, shows on vs off time.
Pressing value encoder changes current digit.

\subsubsection{Amplitude menu}
Shows current amplitude and DC offset.
Pressing menu encoder shows min and max values as well.
Pressing value encoder changes current digit.

\subsubsection{Sig menu}
Shows current signal type and output enable state
Pressing menu encoder TBD.
Pressing value encoder changes current digit.

\pagebreak
\section{Mechanical}
\subsection{Box}
Box could be the \href{https://www.mgelectronic.rs/kutija-plasti%C4%8Dna-z4a-2}{Kradex Z4A} in black. 

\pagebreak
\section{References}
\href{http://www.vwlowen.co.uk/arduino/AD9833-waveform-generator/AD9833-waveform-generator.htm}{AD9833 Waveform Generator by vwlowen}

\end{document}
